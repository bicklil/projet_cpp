\documentclass{report}

\usepackage[utf8]{inputenc}
\usepackage[french]{babel}
\usepackage{geometry}
\usepackage{helvet}
\usepackage{verbatim}
\usepackage{graphicx}

\title{Rapport du Projet de C++}
\author{LEMAN Jean-Christophe, CARON Alexandre, COLLOT Kévin}
\date{\today}

\pagestyle{headings}

\begin{document}
\maketitle
\tableofcontents


\chapter{Introduction}
\section{Présentation du sujet}
Le but de ce projet est de proposer une solution au problème du voyageur de commerce en utilisant un algorithme génétique.
\par Les algorithmes génétiques appartiennent à la famille des algorithmes évolutionnistes et sont basés sur les mécanismes et concepts du vivant comme par exemple les gènes, les chromosomes, la sélection naturelle ou encore les mutations génétiques. Ils permettent d'obtenir des solutions approchées à un problème d'optimisation là ou les méthodes mathématiques ne sont pas suffisamment efficaces.\par 
Le problème du voyageur de commerce en informatique est un problème d'optimisation, dans lequel étant donné un nombre de villes, ainsi que les distances séparant toutes les paires de villes, doit trouver un chemin le plus cours possible passant une et une seule fois par chacune des villes et terminant dans la ville de départ.
\section{Présentation du groupe}

\chapter{Application}
\section{Spécifications de l'application}

\section{Manuel utilisateur}

\chapter{Bilan}
\section{Difficultés rencontrées}

\section{Résultats obtenus}

\section{Perspectives d'évolution}



\end{document}